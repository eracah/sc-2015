\section{Introduction}
\label{sec:intro}

Matrix algorithms are increasingly important in many large-scale data analysis applications.
Essentially, the reason is that matrices (i.e., sets of vectors in Euclidean
spaces) provide a convenient mathematical structure with which to model data
arising in a broad range of applications

In particular, the low-rank approximation to a data matrix $A$ that is provided
by performing a truncated SVD (singular value decomposition)---or PCA
(principal component analysis) or CX/CUR decompositions---is a very complicated
object compared with what is conveniently supported by traditional database
operations~\cite{Skillicorn07}. Recall that PCA finds mutually orthogonal
directions that maximize the variance captured by the factorization, and CX/CUR
provides an interpretable low-rank factorization by selecting a small number of
columns/rows from the original data matrix.  Described in more detail in
Section~\ref{sxn:low-rank-methods}, these low-rank approximation methods are
popular in small- and medium-scale machine learning and scientific data
analysis applications for exploratory data analysis and for
providing compact and interpretable representations of complex matrix-based
data, but their implementation at scale remains a challenge.

These matrix-based methods do not mesh well with much of the work that has been popular recently in large scale data analysis.
For example, MapReduce/Hadoop does a certain amount of reading and writing at every step and thus iterative algorithms are prohibitive~\cite{DG08_CACM}.
Apache Spark solves some of these problems by maintaining additional state, but even there systems are not designed for nontrivial matrix algorithms~\cite{SPARK_NSDI_12}. There is an associated question of the ideal hardware platform for running Big Data Analytics. Conventional EC2 class hardware utilizes loosely coupled nodes; whereas typical HPC system are much more tightly coupled with high performance interconnects. Frameworks such as Hadoop and Spark have been developed for EC2 class hardware, and their performance on HPC hardware, especially for complex analytics problems (including matrix decompositions) is largely unexplored. 

In this paper, we address the following research questions:
\begin{itemize}
\item Can we successfully apply low rank matrix factorization methods (such as CX) to a TB-scale scientific dataset?

\item Can we implement CX in a contemporary data analytics framework such as Spark?

\item What is the performance gap between a highly tuned C, and a Spark-based CX implementation? 

\item How well does a Spark-based CX implementation scale on modern HPC and data-center hardware platforms?
\end{itemize}

We start with a description of matrix factorization algorithms in Section~\ref{sxn:low-rank-methods}, followed by single node and multi-node implementation details in Section~\ref{sec:implementation}. We review the experimental setup for our performance tests in Section~\ref{sec:setup}, followed by results and discussion in Section~\ref{sec:results}.

